% Please use the review version when submitting papers for review.
% The option below provides the final form of your paper

%\documentclass[review,authoryear,jfv]{beg_32}      %  review version
\documentclass[article,authoryear,jfv]{beg_32}             %  final version
% Use option "equation" for numbering equation as section

%\count0=115
\usepackage[hang]{footmisc}
\setlength{\footnotemargin}{0in}
\frenchspacing
\fancypagestyle{plain}{%
  \fancyhf{}
  \fancyhead[R]{\small {\it \jname}, x(x): \thepage--\pageref{LastPage} (\myyear\today)}
  \fancyfoot[R]{\small\bf\thepage }
  \fancyfoot[L]{\fottitle}
  }
%\fancypage{\fbox}{}
\renewcommand{\dmyy}{17}
\renewcommand{\myyear}{2017}
\renewcommand{\today}{}
\begin{document}

\volume{Volume x, Issue x, \myyear\today}
\title{Harnessing Machine Learning for Analyzing
Droplet Evaporation in Phase-Change Heat Transfer}
\titlehead{Small Article }
\authorhead{Author, Author, \& Author}
%For at least  authors with different addresses, use instead the following commands
\corrauthor[1]{First A. Author}
\author[1]{S.B. Author, Jr.}
\author[2]{Third Author}
\corremail{f.author@affiliation.com}
\corraddress{Business or Academic Affiliation 1, City, State, Zip Code}
\address[1]{Business or Academic Affiliation 1, City, State, Zip Code}
\address[2]{Business or Academic Affiliation 2, City, Province, Zip Code, Country Business or Academic Affiliation 2, City, Province, Zip Code, Country}
% End information for at least  authors with different addresses
% For authors with the same post address,
%\corrauthor{First A. Author}
%\corremail{f.author@affiliation.com}
%\author{Second B. Author, Jr.}
%\address{Department of Chemistry and Courant, Institute of Mathematical Sciences, New York, NY 10012, USA}
% End commands for all authors with the same address

%\dataO{mm/dd/yyyy}
%\dataO{}
%\dataF{mm/dd/yyyy}
%\dataF{}

\abstract{The abstract summarizes key findings in the paper and should be a paragraph no more than 250 words.
 The abstract summarizes key findings in the paper and should be a paragraph no more than 250 words.
 The abstract summarizes key findings in the paper and should be a paragraph no more than 250 words.
 The abstract summarizes key findings in the paper and should be a paragraph no more than 250 words.
 The abstract summarizes key findings in the paper and should be a paragraph no more than 250 words.
}

\keywords{select up to 10 key terms for a search on
your document}


\maketitle




\section{Introduction}
Heat is difficult to measure, even at macroscales! At macroscales, key
quantities of interest for heat transfer are temperature, heat flux, and
thermophysical properties such as thermal conductivity, specific heat, etc.
Conductive and convective heat transfer at macroscale are usually governed
by diffusion processes because heat carriers (molecules, electrons, phonons,
etc.) in these processes have short mean free paths and short wavelengths.
In micro- and nanostructures, however, the mean free paths and even
wavelengths of heat carriers become comparable or longer than the
characteristic length involved in the transport process. Heat transfer can no longer be
described by established theories applicable to macroscale. It is precisely
these deviations from continuum that have drawn significant interests from
scientific communities to understand micro-/nanoscale heat transfer.
Such understandings have potential impacts over a wide range of
applications, from microelectronics to energy conversion. Experimentally
probing heat transfer in micro-/nanostructures is essential for scientific
and technological endeavors, and significant progress has been made over the
last two decades. This volume aims to provide a summary of the advances made
in probing micro-/nanoscale heat transfer.
\begin{equation}
  R(E) = \frac{\kappa}{2\eta} \int_{\Omega}\left( \nabla{E}\cdot\nabla{E} +
  \varepsilon \right)^{\eta} \, d\Omega
\end{equation}



In many \textit{gasoline direct injection} (GDI) {\it engines hollow cone} sprays generated by swirl or outwardly opening injectors are applied. In this study the spray of an outwardly opening injector is investigated. According to the geometrical shape of an outwardly opening pintle nozzle the fuel exits from an annular gap. Previous investigations have already shown that the hollow cone spray leaving this nozzle is composed of many single strings instead of a continuous conical sheet.

\section{Problem definition}
\label{sec:ProblemDef} In this section, we follow the notation
in~\cite{Xiang,Ghanem}.  Define a complete probability space
$(\Omega,\mathcal{F},\mathcal{P})$ with sample space $\Omega$ which
corresponds to the outcomes of some experiments,
$\mathcal{F}\subset 2^\Omega$ is the $\sigma$-algebra of subsets in
$\Omega$ (these subsets are called events) and
$\mathcal{P}:\mathcal{F}\rightarrow[0,1]$ is the probability
measure. Also, define $D$ as a $d$-dimensional bounded domain
$D\subset\mathbb{R}^d \ (d=1,2,3)$ with boundary $\partial D$. We
are interested to find a stochastic function $u:\Omega \times D
\rightarrow \mathbb{R}$ such that for $\mathcal{P}$-almost
everywhere (a.e.) $\omega \in \Omega$, the following equation holds:
\begin{equation}
  \mathcal{L}(\mbox{\boldmath $x$},\omega;u) = f(\mbox{\boldmath $x$},\omega),~~\forall \mbox{\boldmath $x$} \in D,
  \label{eqn:SDE}
\end{equation}
\noindent
and
\begin{equation}
  \mathcal{B}(\mbox{\boldmath $x$};u) = g(\mbox{\boldmath $x$}),~~\forall \mbox{\boldmath $x$} \in \partial D,
  \label{eqn:Boundary}
\end{equation}
\noindent
where $\mbox{\boldmath $x$} = (x_1,\dots,x_d)$ are the coordinates
in $\mathbb{R}^d$, $\mathcal{L}$ is (linear/nonlinear) differential
operator, and $\mathcal{B}$ is a boundary operator. In the most
general case, the operators $\mathcal{L}$ and $\mathcal{B}$ as well
as the driving terms $f$ and $g$, can be assumed random. We assume
that the boundary has sufficient regularity and that $f$ and $g$ are
properly defined such that the problem in
Eqs.~(\ref{eqn:SDE})--(\ref{eqn:Boundary}) is well-posed
$\mathcal{P}$ -a.e. $\omega \in \Omega$.




\subsection{The Finite-Dimensional Noise Assumption and the Karhunen-Lo\`eve
Expansion the Finite-Dimensional Noise Assumption and the Karhunen-Lo\`eve
Expansion}\label{sec:KLE} Any second-order stochastic process can be
represented as a random variable at each spatial and temporal
location. Therefore, we require an infinite number of random
variables to completely characterize a stochastic process. This
poses a numerical challenge in modeling uncertainty in physical
quantities that have spatio-temporal variations, hence necessitating
the need for a reduced-order representation (i.e., reducing the
infinite-dimensional probability space to a finite-dimensional one).
Such a procedure, commonly known as a `finite-dimensional noise
assumption' \cite{FooPhD,2006AIPC..845..479B}, can be achieved through any
truncated spectral expansion of the stochastic process in the
probability space. One such choice  is the Karhunen-Lo\`eve (K-L)
expansion \cite{Ghanem}.
\subsubsection{Suspended Structures}

Suspended structures used for nanowire
thermal conductivity measurements serve as a good example.
From the entire volume, it is clear that significant progress has been made
in experimental techniques to probe nanoscale heat transfer phenomena, and
the experimental results have led to new understandings of heat transfer
physics, generated new challenges, and opened new opportunities. From my own
perspective, the following are some significant challenges.

\paragraph{Suspended Structures}

Suspended structures used for nanowire
thermal conductivity measurements serve as a good example.
From the entire volume, it is clear that significant progress has been made
in experimental techniques to probe nanoscale heat transfer phenomena, and
the experimental results have led to new understandings of heat transfer
physics, generated new challenges, and opened new opportunities. From my own\footnote{Suspended structures used for nanowire
thermal conductivity measurements serve as a good example.}
perspective, the following are some significant challenges.

\begin{theorem}\label{un_energy}
There exists a unique solution $u_n \in L^2\left(O, H_0^1\left(D\right)\right)$ to the problem \eqref{eqn:SDE} and \eqref{eqn:Boundary} for $n=0$,
and the problem \eqref{eqn:SDE}--\eqref{un_est} for $n\geq 1$. In addition, if $f\in L^2\left(O, H^{-1+\sigma}\left(D\right)\right)$ for $\sigma\in(0,1]$, it holds that
\begin{equation}\label{un_est}
E(|{u_n}|_{H^{1+\sigma}(D)}^2) \leq  C_0^{n+1} \;E({f}_{H^{-1+\sigma}(D)}^2),
\end{equation}
for some constant $C_0$ independent of $n$ and $s$.
\end{theorem}

%\medskip

\begin{proof}
For $n=0$, the existence of the weak solutions can be deduced from the Lax-Milgram theorem,
and the desired energy estimate,
\begin{equation*}
E(|{u_0}|_{H^{1+\sigma}(D)}^2) \leq  \tilde C_0\;E({f}_{H^{-1+\sigma}(D)}^2),
\end{equation*}
This completes the proof.
\end{proof}





\begin{figure}[!t]
	\centering
  \includegraphics[width=12cm]{Fig1.eps}\\
  (a)\hspace*{200pt}(b)
\caption{Visualization setup. (a) Some description for left part. (b) Some description for right part.}
\label{fig:VisSetup}
\end{figure}
\begin{table}[!b]
\centering\begin{threeparttable}[!b]
\caption{Injected fuel mass}
\centering\begin{tabular}{|ccc|}
\hline
\textbf{Injection duration}   & \textbf{Injection pressure} & \textbf{Fuel mass}  \\
$\Delta t$ \textbf{(\textmu s)} & \textbf{(MPa)} & \textbf{(mg)} \\
\hline
300 & 1 & 1.7 \\
300 & 2 & 3.0 \\
300 & 3 & 4.0 \\
300 & 4 & 4.7 \\
300 & 5 & 5.4 \\
300 & 6 & 6.0 \\
300 & 7 & 6.5 \\
300 & 8 & 6.9 \\
300 & 9 & 7.2 \\
300 & 10 & 7.6 \\
300 & 11 & 7.7 \\
300 & 12 & 8.0 \\
\hline
100 & 10 & 0.6 \\
200 & 10 & 4.3 \\
400 & 10 & 10.3\tnote{a} \\
\hline
\end{tabular}
\small{\begin{tablenotes}[normal,flushleft]
\item [a]\hspace*{-3pt} the first note.
\end{tablenotes}}
\end{threeparttable}
\end{table}

\section{Adaptive sparse grid collocation method (ASGC)}
\label{sec:ASGC} In this section, we briefly review the development
of the ASGC strategy. For more details, the interested reader is
referred to~\cite{Xiang,KlimkePhD}.

The basic idea of this method is to have a finite element
approximation for the spatial domain and approximate the
multi-dimensional stochastic space $\Gamma$ using interpolating
functions on a set of collocation points $\{\bold{Y}_i\}_{i=1}^k \in
\Gamma$.  Suppose we can find a finite element approximate solution
$u$ to the deterministic solution of the problem in
Eq.~(\ref{eqn:SDE}), we are then interested in constructing an
interpolant of $u$ by using linear combinations of the solutions
$u(\cdot,\bold{Y}_i)$. The interpolation is constructed by using the
so called sparse grid interpolation method based on the Smolyak
algorithm. In the context of incorporating adaptivity, we have
chosen the collocation point based on the Newton-Cotes formulae
using equidistant support nodes. The corresponding basis function is
the multi-linear basis function constructed from the tensor product
of the corresponding one-dimensional functions.

\begin{algorithm}[!t]
	\caption{Block-circulant embedding method (BCEM)}
	\label{alg:bcem}
Given $N\in \mathbb{Z}^d, x_\textbf{0}\in \Omega$, and strictly positive valued vector $h \in \mathbf{R}^d$,\\
	%\begin{algorithmic}
	 	\textit{Step 1}. Choose a vector $m\in \mathbb{Z}^d$ such that $m[i] \geq 2 N[i]$ for all $1 \leq i \leq d$. \label{step01}\\
		\textit{Step 2}.  Compute the first block row of the circulant matrix $C$ as described. \label{step02}\\
		\textit{Step 3}.  Compute the block-diagonal matrix $\Lambda = \mathrm{diag}(\Lambda_0, \cdots,\Lambda_{\overline{m}-1})$. \label{step03}
	%\end{algorithmic}
\end{algorithm}



Any function $f \in \Gamma$ can now be approximated by the following
reduced form:
\begin{equation}
 f\big(\mbox{\boldmath $x$},\bold{Y}\big ) = \sum _{\|{\bf i}\|\leqslant N+q} \sum_{\bold {j}}
w_{\bf j}^{\bf i}(\mbox{\boldmath $x$}) \cdot a_{\bf j}^{\bf
i}(\bold{Y}), \label{eqn:approximate}
\end{equation}
\noindent
where the multi-index $\bold{i} = (i_1,\ldots,i_N)\in \mathbb{N}^N $, the multi-index $\bold{j} = (j_1,\ldots,j_N)\in \mathbb{N}^N$ and
$\|\bold{i}\| = i_1+\cdots+i_N$. $q$ is the sparse grid
interpolation level and the summation is over collocation points
selected in a hierarchical framework~\cite{Xiang,dol04mul}. Here, $w_{\bf
j}^{\bf i}$ is the hierarchical surplus, which is just the
difference between the function value at the current point and
interpolation value from the coarser grid. The hierarchical surplus
is a natural candidate for error control and implementation of
adaptivity.

Heat flux cross small structures such as nanowires is very tiny, and
high-sensitivity heat flux meters are needed for thermal measurements.
Similar to macroscale heat flux measurements, nanoscale heat flux
measurements usually require knowing temperature differences between two
points and thermal resistance between the same points, from which the heat
flux can be calculated. The key for small heat flux measurements is to
create structures with large thermal resistance values between the two
temperature measurements points.




\section{Conclusions}

Significant progress has been made in terms of tailoring the radiative
properties with micro-/nanost\-ructured materials. Rapid developments have
been made in fabricating periodic gratings and nanostructured periodic
arrays of metal materials over thin films and multilayers. Magnetic
responses have been demonstrated in the infrared and even visible spectral
regions. Further research is needed to understand the coupling mechanisms
between various modes and localized surface plasmons. While FDTD and RCWA
can be used to calculate the radiative properties for engineered surfaces
with micro-/nanostructures, faster computational algorithms are needed with
complicated structures to assist the design for specific applications. The
optical constants are often different in the nanostructured materials as
compared with the bulk solid. Furthermore, for high-temperature
applications, the chemical and thermal stability must be considered, as well
as the size- and temperature-dependent optical constants of the materials.
\begin{enumerate}
  \item Aligned metallic nanowires may exhibit unique optical and thermal radiative
properties due to the surface-enhanced absorption and scattering,
anisotropic dielectric function, and magnetic resonance (between parallel
wires), and may be applied to diffraction optics as well as IR polarizers
and in the control of optical and radiative properties
  \item Detailed models
considering both surface scattering and volume scattering will allow better
understanding of the radiative transfer in these inhomogeneous structures.
Methods for fabricating more uniform structures in large areas with a high
yield are still needed.
  \item Measurements at longer wavelengths, i.e., mid- to
far-IR, will help in understanding the effective medium behavior as well as
the magnetic response.
\end{enumerate}
VACNT arrays show great promise for radiometric applications as nearly
perfect absorbers and emitters in a broad spectral region. EMT appears to be
able to describe the visible and infrared properties of CNT arrays
reasonably well. Specular and diffuse black materials made of SWCNTs or
MWCNTs should be valuable for space-borne radiometric systems, high-power
laser radiometers, absolute cryogenic radiometers, and infrared calibration
facilities. The chemical stability and mechanical rigidity of these
structures also need to be further investigated.
A challenge that exists in
the materials growth process is how to control the growth conditions so that
arrays with a controllable degree of alignment and surface morphology can be
reproduced.  However, this error indicator is too sharp and may
result in a non-terminating algorithm.






%% The Acknowledgements part is started with the command \acknowledgements;
%% acknowledgements are then done as normal sections before appendix
%% \acknowledgements

\acknowledgements

This research was supported by the Computational Mathematics program of
AFOSR (grant FA9550-07-1-0139).


%% The Appendices part is started with the command \appendix;
%% appendix sections are then done as normal sections and after Acknowledgements
%% \appendix

%% \section{}
%% \label{}

%% References without bibTeX database:

%\begin{thebibliography}{-8}

%% \bibitem must have the following form:

%\small{
%\bibitem{key}

%...

%}

%\end{thebibliography}

%% References with bibTeX database:

\bibliographystyle{Bibliography_Style}

\bibliography{References}
\end{document} 